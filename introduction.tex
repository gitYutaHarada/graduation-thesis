\chapter{はじめに}
\label{ch:intro}

本章では,本研究を始めるにあたっての動機,および本論文の構成を示す.

ほげ、ホゲ、保気、ほげ、ホゲ、保気、ほげ、ホゲ、保気、ほげ、ホゲ、保気、
ほげ、ホゲ、保気、ほげ、ホゲ、保気、ほげ、ホゲ、保気、ほげ、ホゲ、保気、
ほげ、ホゲ、保気、ほげ、ホゲ、保気、ほげ、ホゲ、保気、ほげ、ホゲ、保気、
ほげ、ホゲ、保気、ほげ、ホゲ、保気、ほげ、ホゲ、保気、ほげ、ホゲ、保気、
ほげ、ホゲ、保気、ほげ、ホゲ、保気、ほげ、ホゲ、保気、ほげ、ホゲ、保気、
ほげ、ホゲ、保気、ほげ、ホゲ、保気、ほげ、ホゲ、保気、ほげ、ホゲ、保気、
ほげ、ホゲ、保気、ほげ、ホゲ、保気、ほげ、ホゲ、保気、ほげ、ホゲ、保気、
ほげ、ホゲ、保気、ほげ、ホゲ、保気、ほげ、ホゲ、保気、ほげ、ホゲ、保気、
ほげ、ホゲ、保気、ほげ、ホゲ、保気、ほげ、ホゲ、保気、ほげ、ホゲ、保気、
ほげ、ホゲ、保気、ほげ、ホゲ、保気、ほげ、ホゲ、保気、ほげ、ホゲ、保気、
ほげ、ホゲ、保気、ほげ、ホゲ、保気、ほげ、ホゲ、保気、ほげ、ホゲ、保気、
ほげ、ホゲ、保気、ほげ、ホゲ、保気、ほげ、ホゲ、保気、ほげ、ホゲ、保気、
ほげ、ホゲ、保気、ほげ、ホゲ、保気。

辻は始めた~\cite{bt:tsuji2011}\cite{IPSJ2012Tsuji}。
岡本も続いた~\cite{bt:ryutaro2012}。
宮入は工夫した~\cite{bt:miyairi2013}\cite{ipsj:miyairi2013}。
岡本は修士論文で極めた~\cite{mt:RyutaroOkamoto2014}。
教室でも注目を集めた~\cite{ipsj:RyutaroOkamoto2014}。

Knuth はすごい~\cite{wikipedia:DonaldKnuth}。
Knuth の昔の論文です~\cite{KnuthRao1975}。
Lamport もすごい~\cite{wikipedia:LeslieLamport}。
Lamport の最近の論文です~\cite{Lamport:2015:BHW:2749359.2736348}。

ほげ、ホゲ、保気、ほげ、ホゲ、保気、ほげ、ホゲ、保気、ほげ、ホゲ、保気、
ほげ、ホゲ、保気、ほげ、ホゲ、保気、ほげ、ホゲ、保気、ほげ、ホゲ、保気、
ほげ、ホゲ、保気、ほげ、ホゲ、保気、ほげ、ホゲ、保気、ほげ、ホゲ、保気、
ほげ、ホゲ、保気、ほげ、ホゲ、保気、ほげ、ホゲ、保気、ほげ、ホゲ、保気、
ほげ、ホゲ、保気、ほげ、ホゲ、保気、ほげ、ホゲ、保気、ほげ、ホゲ、保気、
ほげ、ホゲ、保気、ほげ、ホゲ、保気、ほげ、ホゲ、保気、ほげ、ホゲ、保気、
ほげ、ホゲ、保気、ほげ、ホゲ、保気、ほげ、ホゲ、保気、ほげ、ホゲ、保気、
ほげ、ホゲ、保気、ほげ、ホゲ、保気、ほげ、ホゲ、保気、ほげ、ホゲ、保気、
ほげ、ホゲ、保気、ほげ、ホゲ、保気、ほげ、ホゲ、保気、ほげ、ホゲ、保気、
ほげ、ホゲ、保気、ほげ、ホゲ、保気、ほげ、ホゲ、保気、ほげ、ホゲ、保気、
ほげ、ホゲ、保気、ほげ、ホゲ、保気、ほげ、ホゲ、保気、ほげ、ホゲ、保気、
ほげ、ホゲ、保気、ほげ、ホゲ、保気、ほげ、ホゲ、保気、ほげ、ホゲ、保気、
ほげ、ホゲ、保気、ほげ、ホゲ、保気。
ほげ、ホゲ、保気、ほげ、ホゲ、保気、ほげ、ホゲ、保気、ほげ、ホゲ、保気、
ほげ、ホゲ、保気、ほげ、ホゲ、保気、ほげ、ホゲ、保気、ほげ、ホゲ、保気、
ほげ、ホゲ、保気、ほげ、ホゲ、保気、ほげ、ホゲ、保気、ほげ、ホゲ、保気、
ほげ、ホゲ、保気、ほげ、ホゲ、保気、ほげ、ホゲ、保気、ほげ、ホゲ、保気、
ほげ、ホゲ、保気、ほげ、ホゲ、保気、ほげ、ホゲ、保気、ほげ、ホゲ、保気、
ほげ、ホゲ、保気、ほげ、ホゲ、保気、ほげ、ホゲ、保気、ほげ、ホゲ、保気、
ほげ、ホゲ、保気、ほげ、ホゲ、保気、ほげ、ホゲ、保気、ほげ、ホゲ、保気、
ほげ、ホゲ、保気、ほげ、ホゲ、保気、ほげ、ホゲ、保気、ほげ、ホゲ、保気、
ほげ、ホゲ、保気、ほげ、ホゲ、保気、ほげ、ホゲ、保気、ほげ、ホゲ、保気、
ほげ、ホゲ、保気、ほげ、ホゲ、保気、ほげ、ホゲ、保気、ほげ、ホゲ、保気、
ほげ、ホゲ、保気、ほげ、ホゲ、保気、ほげ、ホゲ、保気、ほげ、ホゲ、保気、
ほげ、ホゲ、保気、ほげ、ホゲ、保気、ほげ、ホゲ、保気、ほげ、ホゲ、保気、
ほげ、ホゲ、保気、ほげ、ホゲ、保気。

ほげ、ホゲ、保気、ほげ、ホゲ、保気、ほげ、ホゲ、保気、ほげ、ホゲ、保気、
ほげ、ホゲ、保気、ほげ、ホゲ、保気、ほげ、ホゲ、保気、ほげ、ホゲ、保気、
ほげ、ホゲ、保気、ほげ、ホゲ、保気、ほげ、ホゲ、保気、ほげ、ホゲ、保気、
ほげ、ホゲ、保気、ほげ、ホゲ、保気、ほげ、ホゲ、保気、ほげ、ホゲ、保気、
ほげ、ホゲ、保気、ほげ、ホゲ、保気。

第\ref{ch:rw}章では,本研究で考える○○の概念について述べる.
第\ref{ch:imple}章で高速処理のミソを定義し,
その実装手法を第\ref{ch:eval}章でキモを語り、
第\ref{ch:con}章でまともにまとめる。
