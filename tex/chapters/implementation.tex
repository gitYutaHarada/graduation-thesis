\chapter{実装}
\label{ch:imple}
\quad

アノテーションの定義と,アノテーションの内部データ構造を説明する~\cite{flanagan2008ruby}.

\section{ズズズズズ}
\label{sec:imple_xx}

モゲという概念を用いる.モゲという概念を用いる.モゲという概念を用いる~\cite{goto2010bioruby}.
モゲという概念を用いる.モゲという概念を用いる.モゲという概念を用いる.
モゲという概念を用いる~\cite{matsumoto2002ruby}.モゲという概念を用いる.モゲという概念を用いる.
モゲという概念を用いる.モゲという概念を用いる.モゲという概念を用いる.
モゲという概念を用いる.モゲという概念を用いる~\cite{fulton2006ruby}.

モゲという概念を用いる.モゲという概念を用いる.モゲという概念を用いる.
モゲという概念を用いる.モゲという概念を用いる.モゲという概念を用いる.
モゲという概念を用いる.モゲという概念を用いる.モゲという概念を用いる.
モゲという概念を用いる.モゲという概念を用いる.モゲという概念を用いる.
図~\ref{fig:rectangle}の長方形に角は四つである。
本文中から必ず参照する。オプションについても調べた。
モゲという概念を用いる.モゲという概念を用いる.モゲという概念を用いる.
モゲという概念を用いる.モゲという概念を用いる.モゲという概念を用いる.

PowerPoint で図を作るときは,スライドサイズを先に指定すると余白が少なくなる.

% setsumei
\begin{figure}[htbp]
\centering
\includegraphics[width=10cm]{sample.pdf} 
\caption{PowerPoint 上での矩形}
\label{fig:rectangle}
\end{figure}%

モゲという概念を用いる.モゲという概念を用いる.モゲという概念を用いる.
モゲという概念を用いる.モゲという概念を用いる.モゲという概念を用いる.
モゲという概念を用いる.モゲという概念を用いる.モゲという概念を用いる.

\subsection{野生の豹と飼育された豹}
\label{subsec:imple_yy}

模気という構造を用いる.模気という構造を用いる.模気という構造を用いる~\cite{richardson2008restful}.
模気という構造を用いる.模気という構造を用いる.模気という構造を用いる.
模気という構造を用いる.模気という構造を用いる.模気という構造を用いる.
表~\ref{tab:情テク}は豹ではなく表である。
本文中から必ず参照する。オプションについては解説しない。

% setsumei
\begin{table}[htbp]
 \caption{情報テクノロジー学科線路側研究室○○情報}
 \label{tab:情テク}
 \centering
  \begin{tabular}{l|lr}
  \hline
  研究室 & 場所 & 人数\\
  \hline
  佐久田研 & O-YYY & 5 \\
  原田研 & O-XXX & 7 \\
  D\"urst 研 & O-527 & 3\\
  大原研 & O-ZZZ & 2 \\
  \hline
 \end{tabular}
\end{table}

模気という構造を用いる.模気という構造を用いる.模気という構造を用いる.
模気という構造を用いる.模気という構造を用いる.模気という構造を用いる.
模気という構造を用いる.模気という構造を用いる.模気という構造を用いる~\cite{sasada2005yarv}.
模気という構造を用いる.模気という構造を用いる.模気という構造を用いる.
模気という構造を用いる.模気という構造を用いる.模気という構造を用いる.
模気という構造を用いる.模気という構造を用いる.模気という構造を用いる.
模気という構造を用いる.模気という構造を用いる.模気という構造を用いる.
模気という構造を用いる.模気という構造を用いる.模気という構造を用いる.

\subsection{完全実装}
\label{subsec:imple_zz}

模気という構造を用いる.模気という構造を用いる.模気という構造を用いる.
模気という構造を用いる.模気という構造を用いる.模気という構造を用いる.
模気という構造を用いる.模気という構造を用いる.模気という構造を用いる.
模気という構造を用いる.模気という構造を用いる.模気という構造を用いる~\cite{richardson2008restful}.


\section{ソースコードの裏の裏の裏}
\label{sec:imple_jj}

裏は匠の技を駆使している.裏は匠の技を駆使している.裏は匠の技を駆使している.裏は匠の技を駆使している.裏は匠の技を駆使している.裏は匠の技を駆使している.裏は匠の技を駆使している.裏は匠の技を駆使している.裏は匠の技を駆使している.裏は匠の技を駆使している.裏は匠の技を駆使している.裏は匠の技を駆使している.裏は匠の技を駆使している.裏は匠の技を駆使している.裏は匠の技を駆使している.裏は匠の技を駆使している.裏は匠の技を駆使している.裏は匠の技を駆使している.裏は匠の技を駆使している.

% setsumei
この着想のすべてはソースコード~\ref{src:pb2bmi}で示すことができる.

\begin{figure}[htbp]
\lstinputlisting[caption=\texttt{bmi.rb} (2011年度プログラミング基礎II第八回),label=src:pb2bmi,frame=single,breaklines=true,{language=Ruby}]{../src/sample.rb}
\end{figure}


\section{文献データベースと BibTeX}
\label{sec:imple_database}

% setsumei
文献データベース文献データベース文献データベース文献データベース文献デー
タベース文献データベースサードベース文献データベース文献データベース文献データベース
文献データベース文献データベースファーストベース文献データベース文献デー
タベース文献データベースセカンドベースホームベース文献データベース文献データベース文献データベース
ホームベース文献データベースです。

