\chapter{使用データと変数の構築}

本章では,本研究で使用したデータセットの概要と,機械学習モデルの学習に適した形式に変換するための前処理およびデータ統合の手法について述べる.
本研究の特徴は,警察庁が公開する事故データ単体での分析にとどまらず,「道路交通センサス」および「医療機関データ」を地理情報に基づいて統合し,多角的な視点から特徴量を構築した点にある.

\section{データセットの概要}

本研究では,分析の基礎となる事故情報に加え,事故発生現場の「走行環境」および事故後の「救命アクセスの良否」をモデルに反映させるため,以下の3つのオープンデータを採用した.


\subsection{警察庁交通事故統計データ}

本研究における分析の核となるデータセットは、警察庁が公開する「交通事故統計オープンデータ」\cite{traffic_stats_estat}である。本データセットは、日本国内で発生した交通事故の全件を網羅しており、
2019年1月1日から2024年12月31日までの6年間にわたる約189万件の事故記録が含まれている。
各レコードは1件の交通事故に対応しており、事故の発生日時や場所といった基礎情報から、道路環境、運転者および車両の属性に至るまで、多角的な変数が詳細に記録されている。

\begin{itemize}

    \item \textbf{発生日時・場所・環境}: 
    事故発生の年月日時分に加え、昼夜の別、天候、曜日、祝日の有無が含まれる。場所情報は、都道府県および市区町村コード、地点コード、緯度・経度情報により特定される。
    
    \item \textbf{道路環境・交通規制}: 
    事故現場の物理的な特徴として、道路形状(交差点、単路等)、道路線形、車道幅員、地形、路面状態、中央分離帯や歩車道区分の有無が含まれる。
    また、信号機、一時停止規制(標識・表示)、最高速度規制、ゾーン規制といった交通制御情報も詳細に記録されている。
    
    \item \textbf{当事者および車両属性}: 
    交通事故に関与した双方の当事者(以下、当事者Aおよび当事者Bとする)に関する属性情報である。
    本データセットにおいて、当事者Aは当該事故における過失が最も重い者、当事者Bはその相手方(過失がより軽い者)として定義されている。
    具体的には、各当事者の種別(歩行者、自転車、四輪車等)、年齢、通行の用途に加え、運転していた車両のエアバッグ装備状況、衝突部位、損壊程度などが記録されている。
    
    \item \textbf{事故類型・被害状況}: 
    事故の結末や形態を表す変数である。事故内容(死亡、負傷等)、事故類型(追突、出会い頭等)、衝突地点、各当事者の人身損傷程度、および死者数・負傷者数が記録されている。
    
\end{itemize}

これらの変数は、カテゴリカルデータ(名義尺度)と数値データ(比例尺度)が混在しており、モデルへの入力にあたっては適切な前処理を要する。

\begin{table}[htbp]
  \centering
  \caption{本研究で使用する主な変数一覧}
  \label{tab:variables_summary}
  \begin{tabular}{ll}
    \hline
    カテゴリー & 主な変数名 \\
    \hline
    発生日時・環境 & 発生日時, 昼夜, 天候, 曜日, 祝日, 市区町村コード, 緯度・経度 \\
    道路環境 & 道路形状, 路面状態, 車道幅員, 信号機, 一時停止規制, 速度規制 \\
    当事者・車両 & 当事者種別, 年齢, 用途, エアバッグ装備, 車両損壊程度 \\
    事故結果 & 事故内容, 事故類型, 死者数, 負傷者数, 人身損傷程度 \\
    \hline
  \end{tabular}
\end{table}




\subsection{道路交通センサス(国土交通省)}
国土交通省が実施する「道路交通センサス」は、原則として5年に一度実施される全国的な交通調査です。
この調査は、日本の道路網が現在どのように利用されているか、交通の量的・質的な実態を把握するための最も基礎的なデータソースとなります。
調査項目は多岐にわたりますが、代表的な指標として以下のものが挙げられます。

\begin{itemize}
    \item \textbf{交通量(Traffic Volume)}\\
    観測地点を単位時間内に通過する車両の台数です。基本指標として「平日12時間交通量(7:00~19:00)」および「平日24時間交通量」が用いられます。また、昼間と夜間の交通比率などもここから算出されます。

    \item \textbf{大型車混入率(Large Vehicle Ratio)}\\
    全交通量に占める大型車(バス、トラック、特種車など)の割合です。
    この指標は、物流の動向把握や、騒音・振動といった沿道環境への影響評価、舗装設計における交通荷重の検討などに不可欠です。

    \item \textbf{混雑度(Congestion Degree)}\\
    道路の幾何構造等から決まる「設計交通容量(Capacity)」に対する実際の「交通量(Volume)」の比率であり、一般に $V/C$ 比(Volume / Capacity Ratio)として表されます。
    混雑度が $1.0$ を超える区間は、交通需要が道路の処理能力を上回っており、慢性的な渋滞が発生している状態であることを示唆します。

    \item \textbf{平均旅行速度(Average Travel Speed)}\\
    特定の区間を走行するのに要した平均的な速度です。ピーク時(混雑時間帯)とオフピーク時の速度差を分析することで、ボトルネック箇所の特定や、移動時間の信頼性評価に活用されます。
\end{itemize}

これらの調査結果は、将来の道路整備計画(バイパス建設や車線拡幅など)の策定、交通需要予測モデルの構築、
および費用対効果の検証などの基礎資料として、日本の道路行政において極めて重要な役割を果たしています。


% \noindent
% 本研究では、上記の道路交通センサスの調査結果に基づき、分析モデルの入力変数として以下の特徴量を抽出・算出した。
% 各特徴量の定義および変数名は以下の通りである。

% begin{description}
%     \setlength{\itemsep}{0mm} 
%     \item[\texttt{traffic\_24h}] \textbf{24時間交通量(全体)}\\
%     観測地点における1日(0:00~24:00)の全通過車両台数。交通需要の規模を表す最も基本的な指標である。

%     \item[\texttt{traffic\_24h\_large} / \texttt{traffic\_24h\_small}] \textbf{車種別24時間交通量}\\
%     全交通量を大型車(バス・トラック等)と小型車(乗用車等)に分離したもの。

%     \item[\texttt{congestion\_degree}] \textbf{混雑度}\\
%     前述の通り、設計交通容量に対する交通量の比率($V/C$)を示す。

%     \item[\texttt{large\_vehicle\_rate}] \textbf{大型車混入率}\\
%     全交通量に占める大型車の割合であり、以下の式で算出される。\\
%     $ \text{大型車混入率} = \frac{\texttt{traffic\_24h\_large}}{\texttt{traffic\_24h}} $

%     \item[\texttt{day\_night\_ratio}] \textbf{昼夜率}\\
%     昼間(7:00~19:00)と夜間(19:00~翌7:00)の交通量の比率。生活交通と物流交通の特性分離などに用いられる。

%     \item[\texttt{peak\_ratio}] \textbf{ピーク率}\\
%     24時間交通量に対する、ピーク時間帯(交通量が最大の1時間)の交通量の割合。

%     \item[\texttt{average\_travel\_speed}] \textbf{平均旅行速度}\\
%     混雑時および非混雑時を含む、区間通過の平均速度(km/h)。
% \end{description}

\subsection{医療機関データ(国土数値情報)} 
国土交通省が提供する「医療機関データ」は、地方厚生局のデータを基に、
全国の医療施設(病院・診療所・歯科診療所)の位置情報と属性情報を統一的なフォーマットで整備したデータセットです。 
このデータは、地域における医療供給体制の空間的配置を把握するための最も基礎的な資料であり、都市計画や医療アクセシビリティの評価において広く活用されています。
データの主な構成要素として、以下の指標が挙げられます。

\begin{itemize} 

    \item \textbf{医療機関区分(Institution Classification)}\\
    対象となる施設が「病院(Hospitals)」であるか「診療所(Clinics)」であるかを示す区分です。 
    医療法において、病床数が20床以上の施設を病院、19床以下(または無床)の施設を診療所と定義しており、
    予測モデルにおいて施設の規模や役割(高度医療か、一次医療か)を区別する際の重要なフィルタリング条件となります。
    
    \item \textbf{診療科目(Clinical Departments)}\\
    内科、外科、小児科、産婦人科など、その医療機関が標榜している診療科目の情報です。    
    特定の疾患や対象患者層(例:小児救急、高齢者医療)に焦点を当てた分析を行う際、
    対象となる医療機関を抽出するために用いられます。    
    
    \item \textbf{病床数(Number of Beds)}\\
    その医療機関が有する入院設備の規模を示す指標です。
    「一般病床」、「療養病床」、「精神病床」などの種別ごとにデータが保持されています。
    病床数は医療機関の「収容能力(Capacity)」の代理変数として扱われることが多く、
    重力モデル等を用いた患者流動の推計においては、施設の誘引力を決定するパラメータとして不可欠です。

    \item \textbf{位置情報(Location Coordinates)}\\
    各医療機関の正確な所在地を示す緯度・経度座標です。
    これにより、居住地(需要点)から医療機関(供給点)までのユークリッド距離や道路ネットワーク距離を算出することが可能となり、
    地理的なアクセス性(Accessibility)を定量化する際の根幹データとなります。

\end{itemize}

これらのデータは、地域医療構想における医療資源の偏在分析、救急搬送の最適配置シミュレーション、
および人口減少社会における持続可能な医療ネットワークの構築に向けた基礎資料として、極めて重要な役割を果たしています。



\section{データクリーニングと変数の時間的分類}

本節では,データの品質確保と,本研究の目的である「実運用可能性」を評価するための変数設計について述べる.

\subsection{データの欠損と整合性確認}
まず,データの欠損や形式の整合性を確認した.事前調査の結果,本データセットにおいて欠損値が含まれるレコードは全体の0\%であり,データの完全性が極めて高いことが確認された.したがって,欠損値に対する特別な処理(削除や補完)は不要であると判断し,全レコードを分析に使用した.

\section{特徴量の選定とデータリークの排除}

本研究では,実運用可能な予測モデルを構築するため,モデルへの入力変数を「事故発生時に現場において客観的に確定している事実(Objective Features)」のみに厳しく限定した.
警察庁データセットには多岐にわたる変数が含まれているが,その中には事故の結果として記述された変数や,発生後でなければ判明しない情報が多く含まれている.これらを適切に処理・除外することは,データリーク(Data Leakage)を防ぐ上で最も重要な工程である.

以下に,本研究で採用した変数および意図的に除外した変数の詳細を述べる.

\subsection{客観的観測値(採用した変数)}
事故発生の瞬間,あるいは発生直前の段階で確定しており,通報や現場到着時に即座に入手可能な情報を「客観的観測値」として定義し,モデルの入力変数として採用した.

\begin{itemize}
    \item \textbf{時間・場所情報}:
    時系列的な傾向や地域特性を捉えるため,`発生日時`を年(\texttt{year})・月(\texttt{month})・時間(\texttt{hour})等に分解して使用した.
    また,`緯度`・`経度`情報は,そのままの特徴量として用いるほか,後述する道路交通センサスや医療機関データとの地理的結合(Spatial Join)および地域ID(\texttt{area\_id})の生成に使用した.

    \item \textbf{環境・気象条件}:
    運転操作に影響を与える外的要因として,`天候`,`路面状態`,および`地形`(平坦,急カーブ等)を採用した.

    \item \textbf{道路物理構造}:
    事故現場の恒久的な特徴として,`道路形状`(交差点,単路等),`道路線形`,`車道幅員`,`中央分離帯施設等`の有無,および`信号機`の有無を採用した.

    \item \textbf{交通規制情報}:
    ドライバに課された法的制約として,`ゾーン規制`,`指定/法定速度規制`に加え,当事者A・Bそれぞれに対する`一時停止規制`(標識および表示の有無)を使用した.
\end{itemize}

\subsection{リーク要因となる変数(除外した変数)}
以下の変数は,事故が発生し,警察による実況見分が行われた「後」に確定する情報,または予測の答えそのもの(目的変数)を含むため,学習用データから徹底的に除外した.

\begin{itemize}
    \item \textbf{結果指標(目的変数関連)}:
    `死者数`および`負傷者数`は,目的変数である「死亡事故フラグ(fatal)」の作成に使用した後,説明変数からは削除した.

    \item \textbf{事故類型}:
    「人対車両」「車両相互」といった`事故類型`は,事故の結果として確定する分類であり,発生前の状況説明変数としては不適切であるため除外した.

    \item \textbf{損傷・損壊の程度}:
    `人身損傷程度`,`車両の損壊程度`(大破・中破等),および`車両の衝突部位`は,衝突の激しさ(結果)を直接的に記述しており,これらを入力することは重大なリークとなる.

    \item \textbf{安全装置の作動状況}:
    `エアバッグの装備`および`サイドエアバッグの装備`については,データセットの仕様上,単なる装備の有無だけでなく「作動したか否か」の情報が混在あるいは推測可能となるリスクがある.本研究では安全側(Conservative)の判断として,これらも一律に除外対象とした.
\end{itemize}

\begin{table}[htbp]
  \centering
  \caption{本研究における主な採用変数と除外変数の一覧}
  \label{tab:feature_selection_detail}
  \begin{tabular}{p{0.15\textwidth} p{0.38\textwidth} p{0.38\textwidth}}
    \hline
    \textbf{カテゴリ} & \textbf{採用変数(発生時情報)} & \textbf{除外変数(事故後情報・リーク)} \\
    \hline
    日時・場所 & 年, 月, 時間, 緯度, 経度 & - \\
    環境 & 天候, 路面状態, 地形 & - \\
    道路構造 & 道路形状, 線形, 幅員, 信号機, 中央分離帯 & - \\
    規制 & 速度規制, 一時停止(標識/表示), ゾーン規制 & - \\
    事故形態 & - & \textbf{事故類型} (人対車両など) \\
    損害結果 & - & \textbf{死者数, 負傷者数}, 人身損傷程度, 車両損壊程度, 衝突部位 \\
    車両装備 & (車種, 用途などは採用) & \textbf{エアバッグ装備} (作動情報の混入回避) \\
    \hline
  \end{tabular}
\end{table}



% 以下のセクションでは、データ構成変数を情報の入手時点(Availability)に基づき「事故発生直後に入手可能な情報(Model P用)」と「事後確定情報(Model Q用)」に厳密に区分することについて述べている。
% また、目的変数(死亡事故フラグ)の定義とリーク変数の除外、およびデータ品質管理(異常値除去・欠損補完)の手法について説明している。
データの構成変数については、本研究の目的である「実運用可能性」と「理論的要因分析」を両立させるため、情報の入手時点(Availability)に基づいた厳密な区分を行っている。
まず、事故発生直後の段階で客観的に特定可能な変数群($I_{pre}$)として、
発生日時(年、月、日、時、分)や発生場所の緯度・経度といった時空間情報に加え、天候や路面状態、地形、道路形状(交差点や単路の区別)、
および信号機や一時停止規制の有無といった交通環境情報が挙げられる。これらは、事故現場において即座に観測可能であるため、リアルタイムなリスク予測を行うモデル(Model P)の入力変数として採用される。

一方で、事故の結果として生じた被害状況や、詳細な警察検分を経て初めて確定する変数群($I_{post}$)もデータセットには含まれている。
具体的には、事故類型(人対車両、車両相互等)、当事者の人身損傷主部位、車両の損壊程度、およびエアバッグの作動状況などがこれに該当する。
これらの情報は、事故発生時点では未確定であるため、予測モデルの学習に含めるとデータリークを引き起こす危険性がある。
したがって、本研究ではこれらの変数を予測モデル(Model P)の入力からは厳密に除外する一方、事後的な要因分析を行うモデル(Model Q)および正解ラベルの作成においてのみ利用するという方針をとっている。

分析の目的変数(正解ラベル)となる「死亡事故」の定義については、元データに含まれる「死者数」カラムに基づいている。
具体的には、死者数が1名以上の事故を「死亡事故 ($y=1$)」、死者数が0名の事故を「負傷事故 ($y=0$)」とする2値分類タスクとして設定した。
なお、目的変数の作成に使用した「死者数」およびそれと強い相関を持つ「負傷者数」カラム自体は、モデルが直接的な答えを参照することを防ぐため、学習データから削除されている。

データの品質管理に関しては、警察庁の統計データは一般に高い信頼性を有しているものの、入力ミス等に起因する異常値が含まれる可能性を考慮し、前処理段階でフィルタリングを実施した。
特に、位置情報(緯度・経度)については、日本国内の有効範囲(北緯24度〜46度、東経122度〜146度)を逸脱しているレコードを自動的に除外することで、
後続の空間結合処理や地域クラスタリングにおけるノイズの混入を未然に防いでいる。
欠損値については、データセット全体における発生率は極めて低いものの、数値変数は中央値で、
カテゴリ変数は欠損を示すカテゴリ値で補完するという統一的な処理を適用することで、モデルの堅牢性を確保している。




\section{特徴量エンジニアリング}
 元のデータセットに含まれる変数は,必ずしも機械学習モデルが解釈しやすい形式ではない.そこで,モデルの学習効率と予測精度を向上させるため,以下の3つの観点から特徴量エンジニアリングを実施した.

\subsection{日時データの分解}
 元データの「発生日時」は日付と時刻を含むタイムスタンプ形式であるが,多くの機械学習アルゴリズムはこれを直接扱うことができない.
また,交通事故には「季節による路面状況の変化」や「時間帯による視界の変化」など,周期的な傾向が存在する.
そこで,「発生日時」カラムを「年(year)」「月(month)」「日(day)」「時間(hour)」の4つの数値データに分解した.これにより,モデルは年ごとのトレンド変化や,
季節性・時間帯ごとのリスク変動を個別の特徴量として学習することが可能となる.なお,情報が重複するため,元の「発生日時」カラムは削除した.

\subsection{地理情報のクラスタリング (Geo-Clustering)}
 事故発生地点を示す「緯度」「経度」は,極めて詳細な位置情報を提供する一方で,そのまま連続値としてモデルに入力すると過学習(Overfitting)を引き起こすリスクがある.
特定の交差点の座標そのものを記憶してしまうと,未知の地点での予測能力が低下するためである.
この問題に対処するため,本研究では教師なし学習の一種であるMiniBatchKMeans法を用いて,日本全国の発生地点を50個のエリア(クラスタ)に分割する処理を行った.
各事故データには,緯度・経度の代わりに \texttt{area\_id}(0〜49のカテゴリ変数)を付与した.これにより,詳細すぎる座標情報を「地域特性」という抽象的な概念に変換し,モデルの汎用性を高めている.

\subsection{道路種別の集約}
 元データの「路線コード」は,国道や県道,バイパス区間などを区別するために約6,800種類のユニーク値を持っている.
このようにカテゴリ数が膨大であると,データの希薄性を招き,モデルの学習が不安定になる.また,バイパスの区間番号のような微細な違いは,死亡事故リスクの予測において本質的な意味を持たない場合が多い.
そこで,路線コードの上位桁に基づき,道路を「一般国道」「主要地方道」「高速自動車国道」「市町村道」など,意味のある15種類のカテゴリ(\texttt{road\_type})に集約した.
この次元削減により,情報の粒度を適正化し,道路環境によるリスクの違いをモデルが捉えやすくなるよう設計した.

\section{目的変数の設定と不均衡データ}
 本研究の目的は,ある事故が「死亡事故」に至るか否かを予測することである.前述の通り「事故内容」カラムはリーク変数として削除したが,目的変数(正解ラベル)を作成するために,「死者数」カラムを用いた.
具体的には,死者数が1名以上の事故を「死亡事故($y=1$)」,0名の事故を「非死亡事故($y=0$)」と定義した.
ここで特筆すべきは,クラス間の極端な不均衡である.全データ約189万件のうち,死亡事故は約1.6万件に過ぎず,その比率はおよそ 118:1 となっている.
このような不均衡データ(Imbalanced Data)をそのまま学習させると,モデルはすべてのデータを「非死亡事故」と予測するだけで99\%以上の正解率を出せてしまうため,死亡事故の検知に失敗する.
この問題に対する対策(Class WeightおよびSMOTEの適用)については,次章の実験設定にて詳述する.