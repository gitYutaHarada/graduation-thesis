\chapter{使用データと変数の構築}

本章では,予測モデルの構築に用いたデータセットと,その前処理手法について述べる.
まず3.1節で各データソースの詳細を説明し,続く3.2節以降で具体的な変数化の手順および統合プロセスについて論じる.

\section{データセットの概要}

本研究では,分析の基礎となる事故情報に加え,
事故発生現場の「走行環境」および事故後の「救命アクセスの良否」をモデルに反映させるため,
多角的な視点から以下の3つのオープンデータを採用した.


\subsection{警察庁交通事故統計データ}

本研究における分析の核となるデータセットは,警察庁が公開する「交通事故統計オープンデータ」\cite{traffic_stats_estat}である.
本データセットは,日本国内で発生した交通事故の全件を網羅しており,
2019年1月1日から2024年12月31日までの6年間にわたる約189万件の事故記録が含まれている.
各レコードは1件の交通事故に対応しており,事故の発生日時や場所といった基礎情報から,道路環境,運転者および車両の属性に至るまで,
多角的な変数が詳細に記録されている.

\begin{quote}
\begin{description}
\item[発生日時・場所および環境条件]\mbox{}\\
事故発生の年月日時分に加え,昼夜の別,天候,曜日,祝日の有無が含まれる.場所情報は,都道府県および市区町村コード,地点コード,緯度・経度情報により特定される.

\item[道路環境および交通規制]\mbox{}\\
事故現場の物理的な特徴として,道路形状(交差点,単路等),道路線形,車道幅員,地形,路面状態,中央分離帯や歩車道区分の有無が含まれる.また,信号機,一時停止規制(標識・表示),最高速度規制,ゾーン規制といった交通制御情報も詳細に記録されている.

\item[当事者属性および車両情報]\mbox{}\\
交通事故に関与した双方の当事者に関する属性情報である.本データセットにおいて,当事者Aは当該事故における過失が最も重い者,当事者Bはその相手方(過失がより軽い者)として定義されている.具体的には,各当事者の種別(歩行者,自転車,四輪車等),年齢,通行の用途に加え,運転していた車両のエアバッグ装備状況,衝突部位,損壊程度などが記録されている.

\item[事故類型と被害状況]\mbox{}\\
事故の結末や形態を表す変数であり,事故内容(死亡,負傷等),事故類型(追突,出会い頭等),衝突地点,各当事者の人身損傷程度,および死者数・負傷者数が記録されている.
\end{description}
\end{quote}

これらの変数は,カテゴリカルデータ(名義尺度)と数値データ(比例尺度)が混在しており,モデルへの入力にあたっては適切な前処理を要する.

\begin{table}[tp]
  \caption{本研究で使用する主な変数一覧}
  \label{tab:variables_summary}
  \centering
  \begin{tabular}{l|l}
    \hline \hline
    \multicolumn{1}{c|}{カテゴリー} & \multicolumn{1}{c}{主な変数名} \\
    \hline

    発生日時・環境 & 発生日時, 昼夜, 天候, 曜日, 祝日, 市区町村コード, 緯度・経度 \\
    道路環境 & 道路形状, 路面状態, 車道幅員, 信号機, 一時停止規制, 速度規制 \\
    当事者・車両 & 当事者種別, 年齢, 用途, エアバッグ装備, 車両損壊程度 \\
    事故結果 & 事故内容, 事故類型, 死者数, 負傷者数, 人身損傷程度 \\
    \hline \hline
  \end{tabular}
\end{table}




\subsection{道路交通センサス(国土交通省)}
国土交通省が実施する「道路交通センサス」~\cite{mlit_census}は,原則として5年に一度実施される全国的な交通調査です.
この調査は,日本の道路網が現在どのように利用されているか,交通の量的・質的な実態を把握するための最も基礎的なデータソースとなります.
調査項目は多岐にわたりますが,代表的な指標として以下のものが挙げられます.

\begin{quote}
\begin{description}
\item[交通量]\mbox{}\\
観測地点を単位時間内に通過する車両の台数です.基本指標として「平日12時間交通量(7:00~19:00)」および「平日24時間交通量」が用いられます.また,昼間と夜間の交通比率などもここから算出されます.

\item[大型車混入率]\mbox{}\\
全交通量に占める大型車(バス,トラック,特種車など)の割合です.
この指標は,物流の動向把握や,騒音・振動といった沿道環境への影響評価,舗装設計における交通荷重の検討などに不可欠です.

\item[混雑度]\mbox{}\\
道路の幾何構造(車線数や道幅)に基づいて算出される「理論上の最大交通可能台数」である
「設計交通容量」に対する実際の「交通量」の比率であり,
一般に V /C 比として表されます.
混雑度が $1.0$ を超える区間は,交通需要が道路の処理能力を上回っており,慢性的な渋滞が発生している状態であることを示唆します.

\item[平均旅行速度]\mbox{}\\
特定の区間を走行するのに要した平均的な速度です.ピーク時(混雑時間帯)とオフピーク時の速度差を分析することで,ボトルネック箇所の特定や,移動時間の信頼性評価に活用されます.
\end{description}
\end{quote}

  これらの調査結果は,将来の道路整備計画(バイパス建設や車線拡幅など)の策定,交通需要予測モデルの構築,
  および費用対効果の検証などの基礎資料として,日本の道路行政において極めて重要な役割を果たしています.



  
\subsection{医療機関データ(国土数値情報)} 
国土交通省が提供する「医療機関データ」~\cite{mlit_medical}は,地方厚生局のデータを基に,
全国の医療施設(病院・診療所・歯科診療所)の位置情報と属性情報を統一的なフォーマットで整備したデータセットです. 
このデータは,地域における医療供給体制の空間的配置を把握するための最も基礎的な資料であり,都市計画や医療アクセシビリティの評価において広く活用されています.
データの主な構成要素として,以下の指標が挙げられます.

\begin{quote}
\begin{description}
\item[医療機関区分]\mbox{}\\
対象となる施設が「病院」であるか「診療所」であるかを示す区分です.
医療法において,病床数が20床以上の施設を病院,19床以下(または無床)の施設を診療所と定義しており,~\cite{MedicalCareAct}
予測モデルにおいて施設の規模や役割(高度医療か,一次医療か)を区別する際の重要なフィルタリング条件となります.

\item[診療科目]\mbox{}\\
内科,外科,小児科,産婦人科など,その医療機関が標榜している診療科目の情報です.
特定の疾患や対象患者層(例:小児救急,高齢者医療)に焦点を当てた分析を行う際,
対象となる医療機関を抽出するために用いられます.

\item[病床数]\mbox{}\\
その医療機関が有する入院設備の規模を示す指標です.~\cite{MedicalCareAct}
「一般病床」,「療養病床」,「精神病床」などの種別ごとにデータが保持されています.
病床数は医療機関の「収容能力」の代理変数です.
大規模な施設ほど高度な医療機能を持ち,より広範囲から患者を受け入れる傾向があるため,
本モデルにおいて施設の重要度や誘引力を測るための指標として採用しました.

\item[位置情報]\mbox{}\\
各医療機関の正確な所在地を示す緯度・経度座標です.
これにより,居住地(需要点)から医療機関(供給点)までのユークリッド距離や道路ネットワーク距離を算出することが可能となり,
地理的なアクセス性を定量化する際の根幹データとなります.
\end{description}
\end{quote}

これらのデータは,地域医療構想における医療資源の偏在分析,救急搬送の最適配置シミュレーション,
および人口減少社会における持続可能な医療ネットワークの構築に向けた基礎資料として,極めて重要な役割を果たしています.



\section{データクリーニングと変数の時間的分類}

本節では,データの品質確保と,本研究の目的である「実運用可能性」を評価するための変数設計について述べる.




\subsection{データの欠損と整合性確認}
まず,データの欠損や形式の整合性を確認した.事前調査の結果,本データセットにおいて欠損値が含まれるレコードはなく,
データの完全性が確認された.したがって,欠損値に対する特別な処理(削除や補完)は不要であると判断し,全レコードを分析に使用した.




\subsection{変数の時間的分類とモデル設定}
本研究の主目的は,事故発生直後の限られた情報から死亡事故リスクを予測することにある.
これは,第2章で議論したように,既存の統計分析や事後情報に依存したモデルでは対応不可能な
「リアルタイムの救急リソース配分」や「初期対応の最適化」を実現するためである.
特に,現場到着前の段階で重篤度を推定することは,限られた医療・警察リソースを有効活用し,救命率を向上させる上で不可欠な要件となる.
しかし,警察庁データセット~\cite{traffic_stats_estat}には,事故後の結果として記述された変数や,詳細な現場調査や当事者への聴取を経て初めて確定する情報が多く含まれている.
そこで本研究では,データリークを厳密に管理するため,変数を「情報の可用性」に基づいて以下の2種類に分類した.

\begin{quote}
\begin{description}
\item[事故発生時情報 ($X_{immediate}$)] \mbox{}\\
事故発生の瞬間あるいは直前に客観的に確定している事実(日時,場所,天候,道路構造など)のみとし,事故発生後の詳細な調査によって初めて明らかになる情報(車両の損傷具合,衝突の組み合わせ)は一切含まない.

\item[事故後確定情報 ($X_{post}$)] \mbox{}\\
事故発生後の詳細な実況見分や検分によって初めて確定する情報(詳細な事故パターン,車両の損壊具合,エアバッグが実際に作動したか否かなど)を指す.なお,予測対象である「死亡事故か否か」に直結する「死傷者数」などの結果変数は,データリークとなるため,いずれのモデルにおいても入力変数からは除外している.
\end{description}
\end{quote}

この分類に基づき,本研究では以下の2つの予測モデルを構築する.

\begin{quote}
\begin{description}
    \item[事故発生時情報モデル] \mbox{}\\
    入力変数を $X_{immediate}$ のみに厳しく限定したモデル.実社会でのリアルタイム運用を想定し,データリークとなる $X_{post}$ を徹底的に除外して構築する.
    
    \item[事故後情報モデル] \mbox{}\\
    入力変数として $X_{immediate}$ に加え,$X_{post}$ を使用するモデル.事後分析および理論的な予測精度の上限(参照値)を確認するために構築する.
\end{description}
\end{quote}





\subsection{事故発生時情報モデルにおける変数選定とリークの排除}

「事故発生時情報モデル」の構築にあたり,以下に示す基準で変数の採用・除外を行った.




\subsubsection{採用した変数(客観的観測値)}
現場において客観的に観測可能な変数を,データソースごとに以下のように採用した.

\begin{quote}
\begin{description}
\item[警察庁交通事故統計データ]\mbox{}\\
日時・場所(発生年・月・時,緯度・経度),環境要因(天候,路面状態,地形),道路構造(道路形状,道路線形,信号機の有無,中央分離帯施設等),および交通規制(ゾーン規制,指定/法定速度,一時停止規制の標識・表示)を採用した.

\item[交通量データ]\mbox{}\\
交通量(24時間の全体・大型・小型交通量),交通構成(大型車混入率),時間変動(昼夜率,ピーク率),および走行環境(混雑度,平均旅行速度)を採用した.

\item[医療機関データ]\mbox{}\\
施設へのアクセス(最寄り災害拠点病院までの距離,半径5km圏内の病院数)および施設の規模・機能(最寄り病院の病床数,災害拠点病院フラグ)を採用した.
\end{description}
\end{quote}




\subsubsection{除外した変数(リーク要因の排除)}
以下の変数は,事故の結果そのものであるか,あるいは事故後に確定する情報であるため,$X_{post}$ に該当すると判断し,事故発生時情報モデルの学習データからは除外した.\\

\begin{quote}
\begin{description}
\item[結果指標(目的変数関連)]\mbox{}\\
死者数および負傷者数は,目的変数(死亡事故フラグ)の作成に使用したため,説明変数からは削除した.

\item[事故類型]\mbox{}\\
事故類型(人対車両,車両相互など)は,一見すると発生時の状況に見えるが,詳細な分類は事故後の検分により確定する情報であるため除外した.

\item[損傷・損壊の程度]\mbox{}\\
人身損傷程度,車両の損壊程度,および車両の衝突部位は,衝突の結果を直接的に記述しており,重大なデータリークとなるため除外した.

\item[安全装置の作動状況]\mbox{}\\
エアバッグの装備およびサイドエアバッグの装備については,単なる装備の有無だけでなく「作動したか否か」の情報が含まれる.そのため,本研究ではこれらも一律に除外対象とした.
\end{description}
\end{quote}

以上の処理により,事故発生時情報モデルは,事故発生前のリスク要因のみを純粋に反映した予測が可能となる.

\begin{table}[tp]
  \caption{モデルごとの使用変数セットの定義と目的}
  \label{tab:model_variables}
  \centering
  \begin{tabular}{l|c|c|c}
    \hline \hline
    \multicolumn{1}{c|}{\textbf{モデル名称}} & \multicolumn{1}{c|}{\textbf{発生時情報}} & \multicolumn{1}{c|}{\textbf{事後情報}} & \multicolumn{1}{c}{\textbf{主な目的}} \\
    \multicolumn{1}{c|}{} & \multicolumn{1}{c|}{($X_{immediate}$)} & \multicolumn{1}{c|}{($X_{post}$)} & \multicolumn{1}{c}{} \\
    \hline

    事故発生時情報モデル & $\checkmark$ & --- & リアルタイム予測 \\
    事故後情報モデル & $\checkmark$ & $\checkmark$ & 要因分析 \\
    \hline \hline
  \end{tabular}
\end{table}




\section{特徴量エンジニアリング}
本研究では,警察庁データおよび統合した外部データに対して,モデルの学習効率と予測精度を最大化するために以下の特徴量エンジニアリングを実施した.



\subsection{地理情報のクラスタリングと集約}
緯度・経度の座標情報は粒度が極めて細かく,そのまま連続値としてモデルに入力すると外挿問題や計算コストの増大(高カーディナリティ問題)を招く.
そこで本研究では,MiniBatchKMeansによるクラスタリングと,Geohashによるグリッド集約の2つのアプローチを採用した.



\subsubsection{MiniBatchKMeansによる領域分割}

大規模データに適した \textbf{MiniBatchKMeans ($K=50$)} を用いて,日本全国の発生地点を50個のエリア(クラスタ)に集約した.
本手法は,データをミニバッチ(小集団)単位で逐次処理することで,計算負荷とメモリ消費を大幅に抑制しつつ高速な領域分割を実現するアルゴリズムである.

クラスタ数 $K$ の選定には,エルボー法とシルエット分析を用いた(図\ref{fig:clustering_quality}).
実験の結果,$K=50$ 付近で誤差の減少幅が著しく低下する「エルボー」が確認された.
この値は都道府県数(47)に近く,行政区分にとらわれない物理的な交通特性を反映しつつ,解釈も容易であると判断し採用した.

\begin{figure}[htbp]
  \centering
  \includegraphics[width=1.0\textwidth]{clustering_quality_combined.pdf}
  \caption{クラスタ数 $K$ の変化に伴うSSEとシルエット係数の推移}
  \label{fig:clustering_quality}
\end{figure}

最終的に,各座標を \texttt{area\_id}(0〜49のカテゴリ変数)に変換することで,詳細な位置情報を汎化された「地域特性」としてモデルに取り込んでいる.

\subsubsection{Geohashによる空間集約}

生の緯度経度(連続値)ではなく,Geohashを用いて空間をグリッド(面)として捉えることで,過学習を防ぎつつ近隣エリアのリスクを集約した.

\begin{quote}
\begin{description}
\item[実装内容]\mbox{}\\
\texttt{geohash2} ライブラリを使用し,精度6(約1.2km$\times$0.6km)および精度7(約150m$\times$150m)のグリッドIDを生成した.これにより,モデルは特定の「点」ではなく,一定の広がりを持った「面」としてのリスクを学習可能となる.

\item[履歴特徴量]\mbox{}\\
「そのエリアで過去30日および過去365日に何件の事故が発生したか」を集計し,動的なリスク指標として追加した.

\item[リーク防止策]\mbox{}\\
モデルの学習時に目的変数である「当日の事故情報」が入力特徴量に含まれることを防ぐため,\texttt{.shift(1)} 処理を適用した.これにより,特徴量の算出範囲が厳密に「前日までの観測データ」に限定され,予測時点において利用可能な情報のみに基づいた正当な学習プロセスが保証される.
\end{description}
\end{quote}




\subsection{日時情報の分解とCyclical Encoding}
元データの「発生日時」はタイムスタンプ形式であるが,多くの機械学習アルゴリズムはこれを直接扱うことができない.また,交通事故には「季節による路面状況の変化」や「時間帯による視界の変化」など,周期的な傾向が存在する.
そこで,「発生日時」を以下の2つのアプローチで処理した.

\begin{quote}
\begin{description}
\item[要素分解 (Decomposition)]\mbox{}\\
日時を以下の要素に分解し,個別の特徴量として採用した.
\textbf{年}は長期的なトレンドの学習,
\textbf{月}は季節性の学習,
\textbf{日}は月内の傾向把握,
\textbf{時間}は昼夜や交通量ピークによるリスク変動の学習,
\textbf{曜日}は平日(業務交通中心)と週末(レジャー交通中心)の交通パターンの違いを学習する.

\item[Cyclical Encodingによる時間表現]\mbox{}\\
時間は「23時の次は0時」という連続性を持つが,単純な数値(23 vs 0)ではこの関係性が失われる.これを解決するため,三角関数を用いた円環座標への変換を採用した.
数式は以下の通りである.
\[
x_{\sin} = \sin\left(\frac{2\pi t}{T}\right), \quad x_{\cos} = \cos\left(\frac{2\pi t}{T}\right)
\]
ここで,$t$ は現在の時刻(または月),$T$ は周期(24時間,12ヶ月など)を表す.
この手法の採用理由として,\textbf{連続性の保持}(23時と0時の距離を数学的に近づけ,深夜帯特有のリスクを学習可能にする)と,\textbf{ニューラルネットワークへの適合}(MLPやTabNetにおいて,One-Hot表現よりも密な連続値入力の方が勾配の学習がスムーズに進む)が挙げられる.
\end{description}
\end{quote}

\subsection{道路種別の抽象化}
元データの「路線コード」は,国道や県道,バイパス区間などを区別するために約6,800種類のユニーク値を持っている.このようにカテゴリ数が膨大であると,データの疎性を招き,モデルの学習が不安定になる.
そこで,路線コードの上位桁に基づき,道路を「一般国道」「主要地方道」「高速自動車国道」など,意味のある15種類のカテゴリ (\texttt{road\_type}) に集約した.この次元削減により,情報の粒度を適正化し,道路環境によるリスクの違いをモデルが捉えやすくなるよう設計した.分類の定義を表 \ref{tab:road_type_mapping} に示す.

\begin{table}[tp]
  \caption{路線コードに基づく道路種別の集約定義}
  \label{tab:road_type_mapping}
  \centering
  \begin{tabular}{c|l|l}
    \hline \hline
    \multicolumn{1}{c|}{ID} & \multicolumn{1}{c|}{道路種別} & \multicolumn{1}{c}{路線コード範囲} \\
    \hline

    0 & 一般国道 & 0001 $\sim$ 0999 \\
    1 & 主要地方道・都道府県道 & 1000 $\sim$ 1499 \\
    2 & 主要地方道・市道 & 1500 $\sim$ 1999 \\
    3 & 一般都道府県道 & 2000 $\sim$ 2999 \\
    4 & 一般市町村道 & 3000 $\sim$ 3999 \\
    5 & 高速自動車国道 & 4000 $\sim$ 4999 \\
    6 & 自動車専用道・指定 & 5000 $\sim$ 5499 \\
    7 & 自動車専用道・その他 & 5500 $\sim$ 5999 \\
    8 & 道路運送法上の道路 & 6000 $\sim$ 6999 \\
    9 & 農道 & 7000 $\sim$ 7999 \\
    10 & 林道 & 8000 $\sim$ 8499 \\
    11 & 港湾道 & 8500 $\sim$ 8999 \\
    12 & 私道 & 9000 $\sim$ 9499 \\
    13 & その他 & 9500 \\
    14 & その他の道路 & 9900 \\
    \hline \hline
  \end{tabular}
\end{table}

\subsection{交通量データの整理と特徴量化}
国土交通省「道路交通センサス(一般交通量調査)」のデータを統合し,事故発生地点の道路における交通流の特性を分析に加えた.
本データセットの事故記録には,道路交通センサスの正確な区間ID(リンク番号)が紐付いていないため,以下の共通項目をキーとした確率的なマッピングを用いてデータを統合した.

\begin{quote}
\begin{description}
\item[結合キーの作成]\mbox{}\\
「地域(政令指定都市名または都道府県名)」,「道路種別(15分類)」,および「地形ID(平坦・上り・下り)」の3つの属性を結合キーとして用いた.

\item[集計と結合]\mbox{}\\
センサスデータ側で上記キーが一致する全区間の測定値を算術平均し,それを「その区間の代表的な交通特性」として事故データに結合した.なお,条件が完全に一致する区間が存在しない場合は,より粗い粒度の平均値を用いて補完を行った.
\end{description}
\end{quote}

採用した主な特徴量と,それぞれの追加背景(仮説)および算出ロジックは以下の通りである.

\begin{quote}
\begin{description}
\item[24時間交通量 (\texttt{traffic\_24h})]\mbox{}\\
事故リスクへの「露出量」を表す指標である.交通量が多いほど他車との干渉機会が増加する一方,極端に少ない道路では速度超過による重大事故のリスクが高まるという仮説に基づく.算出ロジックは,該当する区間の平日24時間交通量の算術平均値である.

\item[車種別交通量 (\texttt{traffic\_24h\_large}, \texttt{traffic\_24h\_small})]\mbox{}\\
\texttt{traffic\_24h\_large}(大型車)は「物流・産業利用の強度」を,\texttt{traffic\_24h\_small}(小型車)は「一般車両のアクティビティ(生活・観光利用)」を示す.特に大型車の絶対量は,衝突時の物理的エネルギーが大きいため,死亡事故との相関が高いと想定される.算出ロジックは,平日24時間交通量のうち,大型車および小型車それぞれの算術平均値である.

\item[混雑度 (\texttt{congestion\_degree})]\mbox{}\\
交通流の状態を評価する指標である.値が$1.0$を超える場合は「渋滞(低速・追突多・死亡少)」,低い場合は「自由流(高速・死亡多)」というトレードオフの関係性を検証するために追加した.以下の式で算出される各区間の混雑度の平均値を使用する.
\[ \text{混雑度} = \frac{\text{平日24時間交通量}}{\text{設計交通容量}} \]
(※設計交通容量:その道路区間が物理的に許容できる理論上の最大交通量)

\item[大型車混入率 (\texttt{large\_vehicle\_rate})]\mbox{}\\
全交通量に占める大型車の割合であり,「物理的衝撃の深刻度」の代理変数として用いる.この値が高い道路での事故は,相手方車両へのダメージが甚大になりやすいためである.以下の式により算出する.
\[ \text{大型車混入率} = \frac{\text{平日24時間交通量(大型)}}{\text{平日24時間交通量(全体)}} \]

\item[昼夜率 (\texttt{day\_night\_ratio})]\mbox{}\\
道路の「夜間走行特性」を表す指標である.値が小さいほど夜間の交通割合が高いことを意味し,視認性の低下や居眠り運転によるリスクが増大することを示唆する.昼間(12時間)と夜間(12時間)の交通量の比率として算出する.
\[ \text{昼夜率} = \frac{\text{平日昼間12時間交通量}}{\text{平日夜間12時間交通量}} \]

\item[ピーク率 (\texttt{peak\_ratio})]\mbox{}\\
交通需要の「時間帯集中リスク」を表す.通勤ラッシュ等で特定時間に交通が集中する道路では,急ぎ運転やイライラ運転が誘発されやすいという仮説に基づく.算出ロジックは,ピーク時1時間の交通量が,24時間交通量に占める割合の平均値である.

\item[平均旅行速度 (\texttt{average\_travel\_speed})]\mbox{}\\
混雑度と同様に,実際の交通流の巡航速度レベルを反映したリスク評価を行うために採用した.
\end{description}
\end{quote}

\subsection{医療機関データの整理と特徴量化}
国土数値情報「医療機関データ(救急告示病院)」を使用し,事故発生地点周辺の医療供給体制を評価するための特徴量を構築した.
元データの位置情報は度・分・秒形式であったため,これを十進法に変換した後,BallTreeアルゴリズムを用いて空間的な探索を行った.
BallTreeは,空間データを階層的な球体で管理するデータ構造である.これは2分探索木の概念を空間データに応用したもので,探索点から遠いデータ群をまとめて探索対象から除外(枝刈り)することで,計算量を $O(N)$ から $O(\log N)$ 程度に削減し,高速な近傍探索を実現している.
具体的には,全事故データを対象に以下の2種類の検索を実施した.

\begin{quote}
\begin{description}
\item[最近傍探索 (Nearest Neighbor Search)]\mbox{}\\
各事故地点からHaversine距離(地球を球面と仮定した距離)で最も近い病院を1箇所特定し,その距離と属性を取得した.

\item[周辺探索 (Radius Search)]\mbox{}\\
各事故地点から半径5km以内に存在する病院の個数をカウントした.
\end{description}
\end{quote}

採用した主な特徴量と,それぞれの追加背景(仮説)は以下の通りである.

\begin{quote}
\begin{description}
\item[最寄り病院までの距離 (\texttt{distance\_to\_hospital\_km})]\mbox{}\\
救急医療において,外傷受傷後1時間以内の適切な処置が生存率を左右するという「ゴールデンアワー」の原則に基づく指標である.病院までの物理的距離は搬送時間に直接影響し,治療開始の遅れが致命的となるリスクを捕えるために追加した.算出ロジックは,事故地点(十進法座標)と全病院座標間のHaversine距離(球面距離)に基づき,BallTreeを用いて最近傍探索を行った結果である.

\item[最寄り病院の病床数 (\texttt{nearest\_hospital\_beds})]\mbox{}\\
医療機関の「規模」を表す代理変数である.病床数が多い病院は設備や人員が充実している大規模病院である可能性が高く,その高い処置能力が生存率に寄与すると想定される.特定された最寄り病院に紐付いている詳細データを取得して付与する.

\item[災害拠点病院フラグ (\texttt{nearest\_hospital\_disaster})]\mbox{}\\
「高度医療の可否」を示す指標である.最寄り病院が救命救急センター等の指定を受けているか否かにより,重篤な外傷に対する三次救急医療が受けられる体制の有無を区別する.特定された最寄り病院の施設区分カラムを参照して付与(1: 該当, 0: 非該当)する.

\item[周辺医療資源 (\texttt{hospitals\_within\_5km})]\mbox{}\\
地域における「医療リソースの密度」を表す.選択肢が多い都市部で発生した事故か,医療過疎地で発生した事故かという地理的特性をモデルに反映させるために採用した.事故地点を中心とする半径5kmの範囲に含まれる病院数を,BallTreeの \texttt{query\_radius} メソッドを用いてカウントする.
\end{description}
\end{quote}

同様に,同一の事故地点に対する医療機関データの統合例を以下に示す.

\begin{table}[tp]
  \caption{医療機関データ統合の具体例}
  \label{tab:medical_integration_example}
  \centering
  \begin{tabular}{l|l}
    \hline \hline
    \multicolumn{1}{c|}{\textbf{特徴量}} & \multicolumn{1}{c}{\textbf{値 (意味)}} \\
    \hline
    最寄り病院までの距離 (\texttt{distance\_to\_hospital\_km}) & 0.56 km (非常に近い) \\
    最寄り病院の病床数 (\texttt{nearest\_hospital\_beds}) & 19床 (小規模な施設) \\
    災害拠点病院フラグ (\texttt{nearest\_hospital\_disaster}) & 0 (非該当) \\
    半径5km圏内の病院数 (\texttt{hospitals\_within\_5km}) & 13箇所 (医療密度は高い) \\
    \hline \hline
  \end{tabular}
\end{table}

このデータから読み取れる具体的な状況とその根拠は以下の通りである.

\begin{quote}
\begin{description}
\item[物理的距離(0.56 km)]\mbox{}\\
事故現場から病院までは非常に近く,アクセス自体は良好である.

\item[施設規模(19床・非災害拠点)]\mbox{}\\
最寄りの病院は小規模であり,高度救命処置に対応していない(災害拠点病院フラグが0)可能性が高い.

\item[潜在的リスク]\mbox{}\\
したがって,重篤な事故の場合は近隣の病院を通過し,より遠方の高度医療機関へ搬送する必要が生じる(搬送時間が延びる)リスクが示唆されている.
\end{description}
\end{quote}

\section{データ品質管理と目的変数の設定}

\subsection{データの品質管理 (Quality Control)}
データの信頼性を担保するため,以下の品質管理プロセスを適用した.
\begin{quote}
\begin{description}
\item[座標フィルタリング]\mbox{}\\
元データには稀に誤記による異常な位置情報(例: 緯度経度0.0など)が含まれる場合がある.本研究では日本国内の有効な緯度経度範囲(北緯24度〜46度,東経122度〜146度)を定義し,範囲外のデータを前処理段階で自動的に除去した.除去実績: 全1,895,275件中,168件 (約0.009\%) の異常値を除去し,モデルが異常値に引きずられるリスクを排除した.

\item[欠損値処理]\mbox{}\\
パイプラインにおいて統一的な欠損処理ルールを適用した.数値変数は学習データの中央値 (Median) で補完し,カテゴリ変数は欠損を示す特別値 (\texttt{\_missing}) で補完するよう設計した.ただし,本実験で使用したデータセット(Train/Validation/Test)において欠損値は0件であり,実際にはこの補完処理は発生していない.なお,統合した外部データ(交通量・医療機関)についても,空間結合により全事故データに対して最近傍の情報を付与できたため,欠損値は存在しない.
\end{description}
\end{quote}

\subsection{目的変数の設定と不均衡データ}
本研究の目的は,ある事故が「死亡事故」に至るか否かを予測することである.目的変数 (正解ラベル) は,事故結果の「死者数」に基づいて以下のように定義した.
\begin{quote}
\begin{description}
\item[死亡事故 ($y=1$)]\mbox{}\\
死者数が1名以上の事故

\item[非死亡事故 ($y=0$)]\mbox{}\\
死者数が0名の事故(負傷事故)
\end{description}
\end{quote}

全データ約189万件のうち,死亡事故の割合は約 \textbf{0.86\%} であり,その比率はおよそ \textbf{118:1} という極めて強いクラス不均衡 (Class Imbalance) が存在する.
このような環境下では,全てのデータを「非死亡事故」と予測するだけで99\%以上の正解率が出てしまうため,単純な正解率 (Accuracy) は評価指標として機能しない.
したがって,本研究では不均衡データに適した指標である \textbf{PR-AUC (Precision-Recall Area Under Curve)} を重視してモデルの評価を行う.
これは,モデルが『いかに見逃しなく(再現率)』かつ『誤検知少なく(適合率)』死亡事故を発見できるかを総合的に評価する指標である.本研究のように,『めったに起きないが重要な事象』を予測する際の標準的な評価尺度として採用した.

\section{基礎統計変数の相関とリスク要因}

本研究では,事故発生時点で利用可能な情報のみを用いる「事故発生時情報」と,事故後に判明する情報を含む「事故後情報」の2種類の変数群を定義している.
本節では,モデル構築に先立ち,それぞれの変数群について目的変数(死亡事故フラグ)との関連性を分析した結果を示す.

\subsection{事故発生時情報におけるリスク要因}

事故発生時情報では,事故発生後に判明する情報を除外した変数群を対象とする.
この制約の下でクラメールの連関係数を算出した結果を表\ref{tab:cramers_v_before}に示す.なお,クラメールの連関係数は,カテゴリカル変数(名義尺度)同士の相関の強さを測る指標であり,0から1の範囲をとる.値が大きいほど,その変数が死亡事故の発生と強く関連していることを示す.

\begin{table}[tp]
  \caption{事故発生時情報における特徴量重要度(上位10)}
  \label{tab:cramers_v_before}
  \centering
  \begin{tabular}{c|l|c}
    \hline \hline
    \multicolumn{1}{c|}{順位} & \multicolumn{1}{c|}{特徴量} & \multicolumn{1}{c}{係数 (V)} \\
    \hline

    1 & 当事者種別 & 0.1470 \\
    2 & 道路形状 & 0.0999 \\
    3 & 発生時間帯 (hour) & 0.0774 \\
    4 & 速度規制 & 0.0734 \\
    5 & 道路線形 & 0.0675 \\
    6 & 都道府県コード & 0.0545 \\
    7 & 昼夜 & 0.0536 \\
    8 & 地形 & 0.0518 \\
    9 & 一時停止規制 標識 & 0.0365 \\
    10 & 速度規制 & 0.0321 \\
    \hline \hline
  \end{tabular}
\end{table}

事故後情報が使用できないため,\textbf{道路形状}や\textbf{道路線形},\textbf{発生時間帯}といった\textbf{環境・インフラ情報}が相対的に重要になることが確認された.
また,条件別の死亡率分析(表\ref{tab:fatality_ranking_before})においても,踏切や深夜時間帯といった環境要因が上位を占めている.

\begin{table}[tp]
  \caption{事故発生時情報における死亡事故率が高い条件(抜粋)}
  \label{tab:fatality_ranking_before}
  \centering
  \begin{tabular}{c|c|l|c|c}
    \hline \hline
    \multicolumn{1}{c|}{順位} & \multicolumn{1}{c|}{特徴量} & \multicolumn{1}{c|}{条件内容} & \multicolumn{1}{c|}{死亡率} & \multicolumn{1}{c}{件数(死亡/全)} \\
    \hline

    1 & 当事者種別A & 列車 & 82.38\% & 201 / 244 \\
    2 & 当事者種別A & 特殊車(小型農耕) & 48.52\% & 115 / 237 \\
    3 & 道路形状 & 踏切(第一種) & 30.57\% & 225 / 736 \\
    4 & 当事者種別A & 特殊車(大型その他) & 17.16\% & 64 / 373 \\
    5 & 当事者種別A & 大型二輪(751cc以上) & 10.39\% & 262 / 2521 \\
    6 & 路面状態 & 非舗装 & 7.09\% & 163 / 2300 \\
    7 & hour & 深夜3時台 & 5.38\% & 414 / 7692 \\
    \hline \hline
  \end{tabular}
\end{table}

\subsection{事故後情報におけるリスク要因}

事故後情報では,事故発生後に判明するすべての情報を含めた変数群を対象とする.
この場合のクラメール係数を表\ref{tab:cramers_v_after}に示す.

\begin{table}[tp]
  \caption{事故後情報における特徴量重要度(上位10)}
  \label{tab:cramers_v_after}
  \centering
  \begin{tabular}{c|l|c}
    \hline \hline
    \multicolumn{1}{c|}{順位} & \multicolumn{1}{c|}{特徴量} & \multicolumn{1}{c}{係数 (V)} \\
    \hline

    1 & 事故類型 & 0.1740 \\
    2 & 当事者種別 & 0.1605 \\
    3 & 当事者種別 & 0.1470 \\
    4 & 年齢 & 0.1243 \\
    5 & 道路形状 & 0.0999 \\
    6 & 用途別 & 0.0880 \\
    7 & エアバッグの装備 & 0.0782 \\
    8 & 発生時間帯 (hour) & 0.0774 \\
    9 & 速度規制 & 0.0734 \\
    10 & 道路線形 & 0.0675 \\
    \hline \hline
  \end{tabular}
\end{table}

事故後情報を含めると,\textbf{「事故類型」}(車両単独,列車など)や\textbf{「当事者Bの属性」}が最も強い予測因子として浮上する.
また,条件別の死亡率分析(表\ref{tab:fatality_ranking_after})においては,事故類型「列車」や当事者B情報が上位を占めており,事後情報の予測力の高さが裏付けられた.

\begin{table}[tp]
  \caption{事故後情報における死亡事故率が高い条件(抜粋)}
  \label{tab:fatality_ranking_after}
  \centering
  \begin{tabular}{c|c|l|c|c}
    \hline \hline
    \multicolumn{1}{c|}{順位} & \multicolumn{1}{c|}{特徴量} & \multicolumn{1}{c|}{条件内容} & \multicolumn{1}{c|}{死亡率} & \multicolumn{1}{c}{件数(死亡/全)} \\
    \hline

    1 & 当事者種別A & 列車 & 82.38\% & 201 / 244 \\
    2 & 事故類型 & 列車 & 68.55\% & 255 / 372 \\
    3 & 当事者種別A & 特殊車(小型農耕) & 48.52\% & 115 / 237 \\
    4 & 当事者種別B & 列車 & 41.73\% & 53 / 127 \\
    5 & 道路形状 & 踏切(第一種) & 30.57\% & 225 / 736 \\
    6 & 年齢 & 75歳以上 & 3.11\% & 4682 / 150326 \\
    7 & 事故類型 & 車両単独 & 6.55\% & 4493 / 68580 \\
    \hline \hline
  \end{tabular}
\end{table}

\subsection{分析結果に基づくモデル設計の指針}

以上の分析より,死亡事故リスクは単一の要因ではなく,時間帯・場所・環境といった複数の条件が複雑に絡み合った際に顕在化することが示された.
実際,表\ref{tab:cramers_v_before}に示したように,個々の特徴量の連関係数は最大でも0.15程度と低く,単独の変数だけで死亡事故を決定づける要因が存在しないためである.

また,本分析では「事故後情報(相手方属性や詳細な衝突形態)」が非常に高いリスク説明力を持つことが明らかになった.これは,表\ref{tab:cramers_v_after}において,「事故類型」や「当事者種別B」のクラメール連関係数が,発生時情報の要因よりも高い値を示したことからも裏付けられる.しかし,これらは事故が起きるまで確定しない情報であり,事前の警告には利用できない.
そこで本研究では,あえて入力変数を制限した\textbf{「事故発生時情報モデル」}と,すべての情報を用いた\textbf{「事故後情報モデル」}の2つを構築する.

\begin{quote}
\begin{description}
\item[事故発生時情報モデル(実運用モデル)]\mbox{}\\
事前に入手可能な情報のみを用い,実社会での即応性を確保する.

\item[事故後情報モデル(参照モデル)]\mbox{}\\
事後情報も含めて学習し,到達可能な精度の「上限(理想値)」としてのベンチマークとする.
\end{description}
\end{quote}

両者の性能差(ギャップ)を定量化することで,「現在の観測技術でどこまでリスクを事前に予見可能か」という限界点を明らかにする.

高い予測精度を持つモデルを構築するためには,これらの変数間の非線形な相互作用を捉えることができるアルゴリズムの選定が不可欠である.
次章では,これらの知見に基づいた具体的なモデル構築手法と,不均衡データに対する学習戦略について述べる.

\section{実験設定とデータの分割}

本研究では,過去のデータから未来の事故リスクを予測するというタスクの性質上,評価の公平性を保つためにランダムシャッフルではなく,時系列に基づいたデータ分割を採用した.
全データを発生日時に基づいて以下の3つの期間に分割し,それぞれの役割を明確に定義している.

\begin{quote}
\begin{description}
\item[学習データ (Train): 2019年 〜 2022年(約130.5万件)]\mbox{}\\
モデルのパラメータ学習に使用する.期間は4年間であり,長期的なトレンドや季節性を十分に学習可能な規模を確保している.

\item[検証データ (Validation): 2023年(約30.7万件)]\mbox{}\\
学習時のハイパーパラメータ調整,およびEarly Stopping(早期終了)の判定に使用する.これは,検証データに対する誤差が改善しなくなった時点で学習を打ち切ることで,過学習を未然に防ぐ手法である.

\item[評価データ (Test): 2024年(約28.2万件)]\mbox{}\\
構築したモデルの最終的な性能評価に使用する.このデータは学習プロセス(前処理の統計量算出なども含む)には一切関与させず,正真正銘の未知データとして扱うことで,実運用時の性能を推定する.
\end{description}
\end{quote}
