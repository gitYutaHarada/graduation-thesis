\chapter{関連研究}

本章では,本研究の背景となる既存研究を整理する.まず交通事故の要因分析および死亡事故予測に関するドメイン研究を概観し,次に機械学習を用いた事故予測研究の動向を述べる.
続いて,本研究で重要となる技術的論点──(1)観測時点による特徴量の可用性(事故発生時/事故後)の問題,(2)クラス不均衡に対する手法と評価指標,(3)時系列データにおけるデータリークと分割方法,
および(4)説明可能性と運用化に関する研究──を順に整理し,最後に本研究の位置づけを明確にする.




\section{交通事故死者数削減に向けた社会的要請と予防安全の変遷}
世界的な潮流として,交通事故による死者数をゼロにすることを目指す「Vision Zero」~\cite{Tingvall1999}や,
2020年までに世界の道路交通事故による死傷者数を半減させることを掲げたSDGs(持続可能な開発目標)のターゲット3.6~\cite{UN2015_SDGs}に代表されるように,
交通安全は公衆衛生上の喫緊の課題として位置づけられている.~\cite{WHO2023_GlobalRoadSafety}

これまでの交通安全対策は,
事故多発地点の特定や事後的な道路改良といった「対症療法的なアプローチ」が主であった.\cite{mannering2014analytic}しかし近年では,事故が発生する前の潜在的リスクを評価し,
リソースを重点配分する「予防的・能動的なアプローチ」への転換が求められている.~\cite{itf2016zero}

特に,限られた警察・救急リソースを最適化するためには,単に事故が起きるか否かだけでなく,
その事故が死亡事故に至るかどうかの重篤度を,発生直後の限られた情報から即座に判定する技術が不可欠となっている.
本節では,こうした社会的背景における死亡事故予測の重要性と,従来の事後分析の限界について概観する.




\section{機械学習を用いた事故予測研究の動向}
近年,交通事故予測の領域でも機械学習(ロジスティック回帰,決定木系(Random Forest,GBDT),深層学習など)
を用いた研究が増えている.これらは従来の統計的アプローチ(ポアソン回帰や負の二項分布モデルなど)が変数の線形性を仮定するのに対し,
機械学習モデルは変数間の複雑な非線形相互作用を捉えられるため,予測精度において優位性が示されている.~\cite{Iranitalab2017Comparison}

特にGBDT(LightGBM, XGBoost, CatBoost等)は,交通事故データのような構造化(表形式)データにおいて,
深層学習と同等以上の性能を低い計算コストで発揮するため,多くの先行研究で採用されている.
また,特徴量として位置情報や時間情報を詳述し,空間的・時間的なクラスタリングを行うことで局所的リスクを可視化する研究も報告されている.~\cite{parsa2020toward}

このように,機械学習モデルが交通事故分析において高いポテンシャルを持つことは多くの研究で実証されてきた.
本研究では,これらの機械学習モデル(特にGBDT)をベースとしつつ,これらの成果を実社会での運用(リアルタイムな事故リスク予測や救急搬送支援など)へと昇華させる段階を見据え,
モデルの予測精度だけでなく,入力データの「取得タイミング」という実務的な制約条件に着目する.
次節以降で述べる運用上の課題(事故発生時/事故後区分,不均衡,リーク)を厳密に制御した上で検証を行う.




\section{観測時点による情報の分類と課題}
事故データに含まれる特徴量は,事故発生時点で現場から即時に観測・取得可能な「事故発生時の情報」と,検分や医療記録等,事故発生後に確定する「事故後の情報」に分けられる.

先行研究の多くは,データの豊富さを活かして事故後の情報
(警察庁オープンデータにおける「負傷程度」や「人身損傷主部位」,「車両の損壊程度」等)~\cite{npa_opendata}を含めたモデル構築を行い,高い性能限界を示してきた.~\cite{Alkheder2017}
しかし,これらを事故発生直後のリアルタイム予測に応用しようとする場合,情報の取得タイミングという実運用上の制約が生じる.

本研究では,この制約を考慮し,事故発生時の情報のみを用いたモデルを「事故発生時情報モデル」,
事故後の情報を用いたモデルを「事故後情報モデル」として比較することで,実用性と理論的性能の双方を明確に分離する点に特徴がある.




\section{クラス不均衡への対応と評価指標の選択}
死亡事故は全事故に対して稀な事象であり,この不均衡は学習および評価における重要な課題を生む.
先行研究では,クラス重み付け,オーバーサンプリング(SMOTE等),アンダーサンプリング,Focal Loss,およびアンサンブル手法など多様な対策が提案されている.
評価指標については,陽性クラスが稀な問題設定ではROC-AUCのみでは誤解を生むため,
Precision-Recall 曲線およびPR-AUCがより実用的であると強調される研究が増えている.
また,運用化検討の観点からPrecision@kやRecall@kのような上位k件に対する指標や,
閾値を定めたときの混同行列に基づく運用コスト評価を併用することが推奨されている.




\section{時系列分割・データリークに関する注意点}
交通事故データは発生時刻に伴う強い時系列性を持つため,評価設計には慎重さが求められる.
一般的なランダムな学習/検証分割(k-fold CV等)は,データ間の時間的依存関係を無視し,
未来の情報を学習に利用する「先読みバイアス(Look-ahead Bias)」を引き起こすリスクがある~\cite{Bergmeir2012}.

特に交通事故の発生傾向は,交通流の変化,道路構造の改良,あるいは法規制の変更といった
外部要因により経年的に分布が変化(Concept Drift)する性質を持つ~\cite{Mannering2018}.
そのため,実運用を見据えた評価としては,時間的順序に基づき過去データで学習し未来データで検証する「時系列ホールドアウト」や
「時系列交差検証(Time Series Cross-Validation)」がより適切な手法として推奨されている~\cite{Hyndman2018}.

また,特徴量エンジニアリングの段階におけるデータリーク(Data Leakage)も重要な論点である.
例えば,事故多発地点のエンコーディングやリスク集計において,分割前の全データを用いて統計量を算出してしまうと,
検証データの情報が訓練時に漏洩し,見かけ上の予測性能が著しく過大評価される~\cite{kaufman2012leakage}.
Kaufmanら~\cite{kaufman2012leakage}は,前処理の順序と情報の利用可能性(Availability)を厳密に管理することの重要性を指摘しており,
本研究においてもこの原則(Strict separation of Train/Test calculation)を遵守する必要がある.




\section{説明可能性(Explainability)と運用化の研究}
モデルを実運用に組み込む際には,予測精度の高さに加え,予測の根拠を提示できる「説明可能性」が重要となる.

具体的には,SHAP~\cite{lundberg2017unified}やLIME~\cite{ribeiro2016why}といった特徴量重要度を算出する手法が挙げられる.
これらは,個々の予測事例における特徴量の影響力(局所的な解釈)や,データセット全体での傾向(大域的な解釈)を定量的に示すことができるため,
政策立案者や現場担当者への説明責任を果たす上で有用である.

また,実運用を想定した評価として,閾値設計やDecision Curve Analysis ~\cite{vickers2006decision}への注目も高まっている.
これらは,「予測確率が何%以上なら危険とみなすか」という判定のボーダーライン(閾値)を変化させた際に,
「検知の網羅性」と「誤検知によって生じる無駄なコスト」のバランスを数値化する手法であり,~\cite{elkan2001foundations}
現場のリソース配分を考慮した運用判断において重要な指標となる.これらは,本研究が目指す「実運用性の評価」と密接に関連する.




\section{本研究の位置づけ}
以上の先行研究を踏まえると,交通事故予測研究は機械学習モデルの高度化と説明可能性の導入により一定の成熟をみせている.
しかし,既存研究の多くは予測精度の最大化を主目的としているため,
「観測時点における特徴量の可用性(availability)」という実運用上の制約に着目した体系的な比較検証には,必ずしも十分な焦点が当てられていないのが現状である.

本研究の独自性は,2.3節で定義した実運用を想定した「事故発生時情報モデル」と,
理論的な参照枠となる「事故後情報モデル」を,同一の基準(時系列データ分割,評価指標,前処理手順)の下で構築し,比較評価する点にある.
これにより,事後確定情報を含むモデルが示す「予測性能の理論的上限」に対し,発生直後の情報のみに依存するモデルがどの程度の性能を維持できるかを定量化する.

本研究は,このギャップを明らかにすることで,予測モデルの社会実装における限界を明示するとともに,将来的に優先してリアルタイム化すべき情報の特定に資する知見を提供する.


本章で整理した知見は第4章・第5章のモデル構築および第6章の比較解析で参照される.

