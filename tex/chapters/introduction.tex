\chapter{はじめに}
\label{ch:intro}

% --- 第1段落:現状と背景 ---
 日本国内における交通事故件数は依然として高い水準にあり,喫緊の社会課題となっている.2024年の統計によれば,交通事故発生件数は約29万件,負傷者数は34万人を超え,死者数は2,663名に達している~\cite{traffic_stats_estat}.この死傷者数は,2024年の日本の総人口(約1億2390万人)~\cite{pop_2024}を基に換算すると,およそ357人に1人が1年間のうちに交通事故で死亡または負傷している計算となり,極めて高い確率であると言える.また,交通事故による社会的影響は人的被害にとどまらない.内閣府の調査(2022年度推計)によれば,日本における交通事故起因の経済損失は年間約10兆5,540億円に上るとされており~\cite{econ_loss},これは国民1人あたり年間約8.5万円の負担に相当する.したがって,交通事故の削減は人命を守るだけでなく,社会的な経済負担を軽減する上でも極めて重要である.

% --- 第2段落:問題の複雑性とデータの必要性 ---
 しかしながら,交通事故は単一の要因のみで引き起こされるものではなく,道路環境,気象条件,運転者の属性といった多様な要素が複雑に絡み合って発生する事象である~\cite{traffic_eng_book}.そのため,真に有効な削減策を導出するためには,これらの多次元的な要因を包括的に捉え,その発生メカニズムを解明することが不可欠となる.こうした分析を可能にするリソースとして,近年,警察庁により「交通事故統計オープンデータ」が公開されている~\cite{npa_opendata}.このデータセットは2019年から2024年までの記録を含み,発生日時,天候,道路線形,車両の種類など,事故状況に関する詳細な変数を網羅している.

% --- 第3段落:既存研究と課題(Research Gap) ---
 このオープンデータの公開に伴い,データ活用の試みは徐々に進められている.例えば,自転車関連事故に特化してアソシエーション分析を行った事例~\cite{bicycle_assoc}や,シチズンサイエンスの観点からデータの可視化や市民による分析を推進する試み~\cite{citizen_science}などが報告されている.また,一部では重大事故の要因探索に関する分析~\cite{fuzzy_factor}も行われているが,既存の研究の多くは,特定の車種や対象に限定されているか,あるいは要因の探索的な記述に留まる傾向にある.
 したがって,全国規模の包括的なデータを用い,機械学習によって事故発生リスクを高精度に予測するモデルの構築や,それに基づいた定量的な予防策の提案に至っている事例は未だ限定的である.人間が処理可能な範囲を超えた膨大なデータを活用し,多角的な要因が複合する事故メカニズムを解明する余地が残されている.


% =====  以下今後変更予定 ====

% --- 第4段落:本研究の目的(Purpose) ---
 そこで本研究では,警察庁のオープンデータを用い,機械学習アルゴリズムを活用した死亡事故予測モデルの構築を行う.具体的には,〇〇や〇〇(ここに使う具体的な手法名,例:LightGBMやランダムフォレストなど)等の手法を用いて,事故発生時の状況や環境要因から死亡事故に至るリスクを定量的に評価する.本研究の目的は,予測モデルの解析を通じて死亡事故の主要な要因(Feature Importance)を特定し,交通事故死者数を減少させるためのデータに基づいた客観的な予防策を提言することにある.

% --- 第5段落:論文の構成(Structure) ---
 本論文の構成は以下の通りである.第2章では,本研究で使用する交通事故統計オープンデータの概要および前処理の手順について述べる.第3章では,予測モデルの構築に用いる機械学習アルゴリズムの詳細と評価指標について説明する.第4章では,構築したモデルの精度評価結果および,要因分析から得られた知見について考察を行う.最後に第5章で,本研究の結論と今後の課題についてまとめる.