\chapter{はじめに}

\section{研究の背景}
% --- 第1段落:現状と背景 ---
 日本国内における交通事故件数は依然として高い水準にあり,喫緊の社会課題となっている.2024年の統計によれば,交通事故発生件数は約29万件,負傷者数は34万人を超え,死者数は2,663名に達している~\cite{traffic_stats_estat}.この死傷者数は,2024年の日本の総人口(約1億2390万人)~\cite{pop_2024}を基に換算すると,およそ357人に1人が1年間のうちに交通事故で死亡または負傷している計算となり,極めて高い確率であると言える.また,交通事故による社会的影響は人的被害にとどまらない.内閣府の調査(2022年度推計)によれば,日本における交通事故起因の経済損失は年間約10兆5,540億円に上るとされており~\cite{econ_loss},これは国民1人あたり年間約8.5万円の負担に相当する.したがって,交通事故の削減は人命を守るだけでなく,社会的な経済負担を軽減する上でも極めて重要である.

 しかしながら,交通事故は単一の要因のみで引き起こされるものではなく,道路環境,気象条件,運転者の属性といった多様な要素が複雑に絡み合って発生する事象である~\cite{traffic_eng_book}.そのため,真に有効な削減策を導出するためには,これらの多次元的な要因を包括的に捉え,その発生メカニズムを解明することが不可欠となる.こうした分析を可能にするリソースとして,近年,警察庁により「交通事故統計オープンデータ」が公開されている~\cite{npa_opendata}.このデータセットは2019年から2024年までの記録を含み,発生日時,天候,道路線形,車両の種類など,事故状況に関する詳細な変数を網羅している.

\section{関連研究と本研究の立ち位置}
% --- 第3段落:既存研究と課題(Research Gap) ---
 このオープンデータの公開に伴い,データ活用の試みは徐々に進められている.例えば,自転車関連事故に特化してアソシエーション分析を行った事例~\cite{bicycle_assoc}や,シチズンサイエンスの観点からデータの可視化や市民による分析を推進する試み~\cite{citizen_science}などが報告されている.また,一部では重大事故の要因探索に関する分析~\cite{fuzzy_factor}も行われているが,既存の研究の多くは,特定の車種や対象に限定されているか,あるいは要因の探索的な記述に留まる傾向にある.

 したがって,全国規模の包括的なデータを用い,機械学習によって事故発生リスクを高精度に予測するモデルの構築や,それに基づいた定量的な予防策の提案に至っている事例は未だ限定的である.人間が処理可能な範囲を超えた膨大なデータを活用し,多角的な要因が複合する事故メカニズムを解明する余地が残されている.

\section{本研究の目的と構成}
% --- 第4段落:本研究の目的(Purpose) ---
 そこで本研究では,警察庁のオープンデータを用い,機械学習アルゴリズムを活用した死亡事故予測モデルの構築を行う.具体的には,勾配ブースティング決定木の一種である\textbf{LightGBM(Light Gradient Boosting Machine)}を採用し,ロジスティック回帰やランダムフォレストといった他の手法との比較検証を通じて,不均衡な事故データに対しても高精度な予測が可能なモデルを構築する.
 さらに本研究の特筆すべき目的は,単なる予測にとどまらず,モデルの解釈性(Explainability)を担保することにある.\textbf{SHAP(SHapley Additive exPlanations)値}を用いた要因分析を行うことで,複雑に絡み合う事故要因を定量的に可視化し,「どのような条件下で死亡事故リスクが増大するか」を明らかにする.最終的には,これらの分析結果に基づき,交通事故死者数を減少させるためのデータ駆動型の客観的な予防策を提言することを目指す.

% --- 第5段落:論文の構成(Structure) ---
 本論文の構成は以下の通りである.第2章では,本研究で使用する交通事故統計オープンデータの概要および前処理の手順について述べる.第3章では,予測モデルの構築に用いる機械学習アルゴリズムの詳細と評価指標について説明する.第4章では,構築したモデルの精度評価結果および,要因分析から得られた知見について考察を行う.最後に第5章で,本研究の結論と今後の課題についてまとめる.