\chapter{はじめに}
\label{ch:intro}


日本国内における交通事故件数は依然として高い水準にあり、喫緊の社会課題となっている。2024年の統計によれば、交通事故発生件数は約29万件、負傷者数は34万人を超え、死者数は2,663名に達している~\cite{traffic_stats_estat}。この死傷者数は、2024年の日本の総人口(約1億2390万人)~\cite{pop_2024}を基に換算すると、およそ357人に1人が1年間のうちに交通事故で死亡または負傷している計算となり、極めて高い確率であると言える。また、交通事故による社会的影響は人的被害にとどまらない。内閣府の調査(2022年度推計)によれば、日本における交通事故起因の経済損失は年間約10兆5,540億円に上るとされており~\cite{econ_loss}、これは国民1人あたり年間約8.5万円の負担に相当する。したがって、交通事故の削減は人命を守るだけでなく、社会的な経済負担を軽減する上でも極めて重要である。

しかしながら、交通事故は単一の要因のみで引き起こされるものではなく、道路環境、気象条件、運転者の属性といった多様な要素が複雑に絡み合って発生する事象である~\cite{traffic_eng_book}。そのため、真に有効な削減策を導出するためには、これらの多次元的な要因を包括的に捉え、その発生メカニズムを解明することが不可欠となる。こうした分析を可能にするリソースとして、近年、警察庁により「交通事故統計オープンデータ」が公開されている~\cite{npa_opendata}。このデータセットは2019年から2024年までの記録を含み、発生日時、天候、道路線形、車両の種類など、事故状況に関する詳細な変数を網羅している。
 これほど詳細かつ大規模なデータが存在するにもかかわらず、既存の研究は特定の地域や要因に焦点を当てたものが多く~\cite{existing_study_a, existing_study_b}、包括的な学術的分析や実務的な予防策への適用事例は未だ限定的であるのが現状である。人間が処理可能な範囲を超えた……

 そこで本研究では、警察庁のオープンデータを用いた交通事故予測モデルの構築を行う。機械学習等の予測モデルを用いることで、特定の条件下(例:週末の夜間や特定の道路形状など)におけるリスクを定量的に評価することが可能となる。本研究の目的は、これらの分析を通じて死亡事故につながる主要な危険因子を特定し、交通事故死者数を減少させるための効果的かつ具体的な予防策を導出することにある。