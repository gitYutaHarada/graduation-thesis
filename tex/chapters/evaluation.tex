\chapter{評価}
\label{ch:eval}

\quad


\section{箇条書きの回避}
\label{sec:eval_xxx}

モゲという概念を用いる.モゲという概念を用いる.モゲという概念を用いる.
モゲという概念を用いる.モゲという概念を用いる.モゲという概念を用いる.
モゲという概念を用いる.モゲという概念を用いる.モゲという概念を用いる.
モゲという概念を用いる.モゲという概念を用いる.モゲという概念を用いる.


\begin{itemize}
 \item 1991 年の×△△△△の AAA は秀逸であると感じた.
 \item 2001 年の△△△△△の BBB がよかった.
 \item 1997 年の△◯×の CCC がややよかった.
\end{itemize}

箇条書きは便利ですが,○○なので控える.
論理的なごまかしや妥協につながる可能性があるかもしれないかもしれないと思
うのかもしれない.表や地の文で書く.


モゲという概念を用いる.モゲという概念を用いる.モゲという概念を用いる.
モゲという概念を用いる.モゲという概念を用いる.モゲという概念を用いる.
モゲという概念を用いる.モゲという概念を用いる.モゲという概念を用いる.

\subsection{実検実権実験}
\label{sec:eval_xxxx}

模気という構造を用いる.模気という構造を用いる.模気という構造を用いる.
模気という構造を用いる.模気という構造を用いる.模気という構造を用いる.
模気という構造を用いる.模気という構造を用いる.模気という構造を用いる.
模気という構造を用いる.模気という構造を用いる.模気という構造を用いる.
模気という構造を用いる.模気という構造を用いる.模気という構造を用いる.
模気という構造を用いる.模気という構造を用いる.模気という構造を用いる.
模気という構造を用いる.模気という構造を用いる.模気という構造を用いる.

模気という構造を用いる.模気という構造を用いる.模気という構造を用いる.
模気という構造を用いる.模気という構造を用いる.模気という構造を用いる.
模気という構造を用いる.模気という構造を用いる.模気という構造を用いる.
模気という構造を用いる.模気という構造を用いる.模気という構造を用いる.
模気という構造を用いる.模気という構造を用いる.模気という構造を用いる.
模気という構造を用いる.模気という構造を用いる.模気という構造を用いる.
模気という構造を用いる.模気という構造を用いる.模気という構造を用いる.

\subsection{実験}
\label{sec:eval_y}

模気という構造を用いる.模気という構造を用いる.模気という構造を用いる.
模気という構造を用いる.模気という構造を用いる.模気という構造を用いる.
模気という構造を用いる.模気という構造を用いる.模気という構造を用いる.
模気という構造を用いる.模気という構造を用いる.模気という構造を用いる~\cite{richardson2008restful}.

\section{庭での実験}
\label{sec:eval_z}

特別な庭で実験した.特別な庭で実験した.特別な庭で実験した.特別な庭で実
験した.特別な庭で実験した.特別な庭で実験した.特別な庭で実験した.特別
な庭で実験した.特別な庭で実験した.特別な庭で実験した.特別な庭で実験し
た.特別な庭で実験した.特別な庭で実験した.特別な庭で実験した.特別な庭
で実験した.特別な庭で実験した.

結果は 100 である.結果は 100 である.結果は 100 である.結果は 100 であ
る.結果は 100 である.結果は 100 である.結果は 100 である.結果は 100
である.結果は 100 である.結果は 100 である.結果は 100 である.結果は
100 である.結果は 100 である.結果は 100 である.結果は 100 である.結
果は 100 である.結果は 100 である. 


